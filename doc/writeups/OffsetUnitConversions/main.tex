% arara: pdflatex
% arara: pdflatex
\documentclass[letterpaper,10pt]{article}
\usepackage{amsmath}
\usepackage{amsfonts}
\usepackage{amssymb}
\usepackage{graphicx}
\usepackage{siunitx}
\usepackage{physics}
\usepackage[left=1in,right=1in,top=1in,bottom=1in]{geometry}

\DeclareSIUnit \inch {in}
\DeclareSIUnit \fahrenheit {\degree F}
\DeclareSIUnit \rankine {R}

\title{Offset Unit Conversions}
\author{C.D. Clark III}


\begin{document}
\maketitle

An offset unit is a unit that contains an offset from zero (a bias). The most common examples are the Celsius and Fahrenheit
scales. Converting from Celsius to Kelvin is simple because they have the same
scaling factor, you just add or subtract the offset. However, converting from
Celsius to Fahrenheit requires correct conversion of the scaling factor and offset.

Any non-base unit ($\hat{v}$) can be expressed in terms of a base unit ($\hat{u}$) with a scaling factor ($\alpha$) and offset ($\beta$).
$$
\label{eq:offset_unit}
\hat{v} = \alpha \hat{u} + \beta
$$
There are two questions:
\begin{enumerate}
        \item How do we convert a quantity from one offset unit to another?
        \item How do we convert one offset unit to another offset unit?
\end{enumerate}
% Consider two offset units, $\hat{v}_1$ and $\hat{v}_2$, and assume our base unit $\hat{u}$ is an \emph{abolute} unit.
% \begin{align}
%   \hat{v}_1 &= \alpha_1 \hat{u} + \beta_1 \\
%   \hat{v}_2 &= \alpha_2 \hat{u} + \beta_2
% \end{align}
% We want to know how to convert quantities expressed in one offset unit to another offset units.

It turns out that quantities and units act like covariant and contra-variant vectors, and this has to be
accounted for to derive the correct conversion formula for \emph{units}.
Take the well known conversion formula for Celsius to Fahrenheit:
$$
\label{eq:cel_to_far}
\si{\fahrenheit} = \frac{9}{5} \si{\celsius} + 32.
$$
This is not actually a formula for the units, it is a formula for the components (see below).

First consider normal units with no offset, such as length. For each dimension, we have a base unit, whose scale is equal to one. In the SI system,
our base unit for length is meter, and we can write all other length units as a meter multiplied by a scaling factor,
\begin{align*}
        \si{\centi\meter} &= \SI{0.01}{\meter} \\
        \si{\kilo\meter} &= \SI{100}{\meter}
\end{align*}
In general then, a unit $\hat{u}$ can be written as a base unit $\hat{b}$ times a scaling factor $\alpha$,
\begin{align}
        \hat{u} = \alpha\hat{b}
\end{align}
Different units have different scaling factors,
\begin{align}
        \hat{u}_1 &= \alpha_1\hat{b} \\
        \hat{u}_2 &= \alpha_2\hat{b}.
\end{align}

Any quantity, $\vec{q}$, can be written as a unit, multiplied by a value $q$. But, the quantity's value will depend on the unit, so
\begin{align}
        \label{eq:quanity_as_vec}
        \vec{q} &= q_1 \hat{u}_1 =  q_2 \hat{u}_2.
\end{align}
So, as the notation suggests,  we can think of quantities as vectors, units as basis vectors, and the value of a quantity as it's component with respect to a specific unit. For example,
$l = \SI{2}{\meter} = \SI{200}{\centi\meter}$ is a length quantitiy whose component is 2 in the meter basis and 200 in the centimeter basis.

So, given the component of a quantity with respect to a unit $\hat{u}_1$, what is its component with respect to a unit $\hat{u}_2$? Well,
if we know each units scale factor, then
\begin{align}
        q_2 \hat{u}_2 &=  q_1 \hat{u}_1 = q_1 \alpha_1 \hat{b} = q_2 \alpha_2 \hat{b}, \\
        q_2 \alpha_2 &= q_1 \alpha_1, \\
        q_2 &= \frac{\alpha_1}{\alpha_2} q_1
\end{align}
So, the component for $\hat{u}_2$ is the compoenent for $\hat{u}_1$, multiplied by the ratio of scale factors. This is just the familiar unit convertion method of ``multily by 1''.
\begin{align}
        \hat{u}_1 &= \si{\centi\meter} \\
        \alpha_1 &= 0.01 \\
        \hat{u}_2 &= \si{\kilo\meter} \\
        \alpha_2 &= 1000 \\
        \vec{q} &= \SI{2500}{\centi\meter} = \SI{2500}{\centi\meter} \frac{\SI{1}{\meter}}{\SI{100}{\centi\meter}}\frac{\SI{1}{\kilo\meter}}{\SI{1000}{\meter}} = \frac{0.01}{1000} 2500 \si{\kilo\meter}
\end{align}
Note that this transformation rule is opposite to the rule for the units themselves,
\begin{align}
        \hat{u}_1 = \alpha_1 \hat{b} \\
        \hat{u}_2 = \alpha_2 \hat{b} \\
        \frac{\hat{u}_1}{\alpha_1} = \frac{\hat{u}_2}{\alpha_2}\\
        \hat{u}_2 = \frac{\alpha_2}{\alpha_1} \hat{u}_1
\end{align}
Thats because the units themselves transform like covectors, and the components of quantity's transform like contravariant vectors.

Now consider an offset unit. We can write any offset unit as an absolute base unit multiplied by a scaling factor, \emph{plus} an
offset vector, which can also be written as the base unit multiplied by a scaling factor $\beta$:
\begin{equation}
        \hat{u} = \alpha \hat{b} + \beta \hat{b}
\end{equation}
However, when we use an offset unit to express a quantity, the quantity's value only scales the first term,
\begin{equation}
        \vec{q} = q \hat{u} = q \alpha \hat{b} + \beta \hat{b}.
\end{equation}
Here, the offset is expressed with respect to the base unit $\hat{b}$, but we could express it with respect to the scaled base unit
\begin{equation}
        \hat{u} = \alpha \hat{b} + \beta^\prime \alpha \hat{b}.
\end{equation}
The difference will be the that the numerical value stored for the unit will be different. We have chosen to use the first form because it is
consistent across units, i.e. it will always represent the unit's zero in terms of the base unit. So, if we use Kelvin as our base unit,
then the offset scaling factor for Celsius and Fahrenheit will be 273.15 and 255.37 because $\SI{0}{\celsius} = \SI{273.15}{\kelvin}$ and $\SI{0}{F\degree} = \SI{255.37}{\kelvin}$.

Given the component of a quantity in one unit, what will the component in another unit be? We have
\begin{align}
        \vec{q} &= q_1 \hat{u}_1 = q_1 \alpha_1 \hat{b} + \beta_1 \hat{b} = q_2 \hat{u}_2 = q_2 \alpha_2 \hat{b} + \beta_2 \hat{b} \\
        q_1 \alpha_1 + \beta_1 &= q_2 \hat{u}_2 = q_2 \alpha_2  + \beta_2 \\
        q_2 &= \frac{\alpha_1 q_1 + \beta_1 - \beta_2}{\alpha_2}
\end{align}
So, given an offset unit that is expressed in terms of an absolute base unit, we can convert a quantity to another offset unit expressed in the same base.
For example,
converting \SI{100}{\celsius} to \si{\degree F}
\begin{align}
        \vec{q} &= \SI{100}{\celsius} \\
        q_1 &= 100 \\
        \alpha_1  &= 1 \\
        \beta_1 &= 273.15 \\
        \alpha_2 &= \frac{5}{9} = 0.5555 \\
        \beta_2 &= 255.37 \\
        q_2 &= \frac{ 1\times 100 + 273.15 - 255.37}{5/9} = 212
\end{align}
Finally, we derive how to compute the scaling factors $\alpha$ and $\beta$ for a unit, if its relationship to another unit is known. Here is where we need to be careful because
the familiar unit conversion formula for temperature are actually for the components, \emph{not} the units. So $\si{\degree F} = \frac{9}{5} \si{\celsius} + 32$ is actually
a relationship between $q_\text{F}$ and $q_\text{C}$. This is why it looks like we are adding something with units, $\frac{9}{5} \si{\celsius}$, to something without units $32$,
but really, $\si{\celsius}$ just represents a number here. What we really have is $q_2 = Aq_1 + B$. It then follows that
\begin{align}
        q_2 &= \frac{\alpha_1 q_1 + \beta_1 - \beta_2}{\alpha_2} = Aq_1 + B \\
A &= \frac{\alpha_1}{\alpha_2} \\
B &= \frac{\beta_1 - \beta_2}{\alpha_2} \\
\alpha_2 &= \frac{\alpha_1}{A} \\
\beta_2 &= \beta_1 - \alpha_2 B
\end{align}
This allows us to compute the scaling factors for a new unit if the conversion formula for its \emph{components} are known, and we can use this to derive the correct conversion formula
for the units.
However,
we have to assume some special algebra...
\begin{align}
        x \times (\alpha \hat{b} + \beta \hat{b}) = A\alpha \hat{b} + \beta \hat{b} \\
        (\alpha \hat{b} + \beta \hat{b}) + x = A\alpha \hat{b} + (\beta + x) \hat{b} \\
\end{align}
Then,
\begin{align}
        T_\text{F} &= \frac{9}{5} T_\text{C} + 32 \\
        A &= \frac{9}{5} \\
        B &= 32 \\
        \si{\degree F} &= \frac{5}{9} \si{\celsius} - \frac{5}{9} \times 32 
\end{align}
The library implements the multiply and add operators for units and numbers this way to allow units to be defined naturally in terms of other units.

%\section{Non-linear Transformations}

%In general, a unit may be defined as an arbitrary transformation. For example, the American Wire Gauge unit is defined as
%$$
%d_n = \SI{0.127}{\milli\meter} \times 92^{\frac{36 - n}{39}}
%$$
%where $n$ is the wire gauge. To perform arbitrary unit conversions, including conversions between non-linear units, we just need the
%function that relates unit to a linear, absolute base unit. Given two units:
%\begin{align}
%\hat{v}_1 &= f(\hat{u}) \\
%\hat{v}_2 &= g(\hat{u})
%\end{align}
%%then
%%\begin{align}
%  %\hat{u} &= f^{-1}(\hat{v}_1) \\
%%\hat{v}_2 &= g(f^{-1}(\hat{v}_1)).
%%\end{align}
%To convert quantities, be just convert to the base unit first, then to the second unit.
%\begin{align}
%v_1 \hat{v}_1 &= g(f^{-1}(v_1)) \hat{v}_2 = v_2 \hat{v}_2 \\
%v_2 &= g(f^{-1}(v_1))
%\end{align}


\end{document}

