\documentclass[letterpaper,10pt]{article}
\usepackage{amsmath}
\usepackage{amsfonts}
\usepackage{amssymb}
\usepackage{graphicx}
\usepackage{siunitx}
\usepackage{physics}
\usepackage[left=1in,right=1in,top=1in,bottom=1in]{geometry}

\DeclareSIUnit \inch {in}
\DeclareSIUnit \fahrenheit {\degree F}
\DeclareSIUnit \rankine {R}

\title{Offset Unit Conversions}
\author{C.D. Clark III}


\begin{document}
\maketitle

An offset unit is a unit that contains an offset from zero (a bias). The most common examples are the Celsius and Fahrenheit
scales. Converting from Celsius to Kelvin is simple because they have the same
scaling factor, you just add or subtract the offset. However, converting from
Celsius to Fahrenheit requires correct conversion of the scaling factor and offset.

Any non-base unit ($\hat{v}$) can be expressed in terms of a base unit ($\hat{u}$) with a scaling factor ($\alpha$) and offset ($\beta$).
$$
\label{eq:offset_unit}
\hat{v} = \alpha \hat{u} + \beta
$$
Consider two offset units, $\hat{v}_1$ and $\hat{v}_2$, and assume our base unit $\hat{u}$ is an \emph{abolute} unit.
\begin{align}
  \hat{v}_1 &= \alpha_1 \hat{u} + \beta_1 \\
  \hat{v}_2 &= \alpha_2 \hat{u} + \beta_2
\end{align}

We want to know how to convert quantities expressed in one offset unit to another offset units.

\section{Incorrect Way}
What follows is not correct, but it illustrates the difficulty in deriving a formula to convert from on offset unit to the other.

To convert \emph{from} $\hat{v}_1$ \emph{to} $\hat{v}_2$,
\begin{align}
  \hat{u} &= \frac{\hat{v}_1 - \beta_1}{\alpha_1} \\
  \hat{v}_2 &= \alpha_2 \left( \frac{\hat{v}_1 - \beta_1}{\alpha_1} \right)  + \beta_2 \\
  \label{eq:v1_to_v2}
  \hat{v}_2 &= \frac{\alpha_2}{\alpha_1}\hat{v}_1  + \left( \beta_2- \frac{\alpha_2}{\alpha_1}\beta_1 \right)
\end{align}
If $\beta_1 = \beta_2 = 0$ this just gives the usual unit convertion formula.

Now, given a quantity $x = v_1 \hat{v}_1$, what is the numerical value for the quantity when expressed in terms of $\hat{v}_2$?

\textbf{tangent}
Note that \ref{eq:v1_to_v2} gives us a way to replace $\hat{v}_2$ with
$\hat{v}_1$. Given a units $\hat{v}_1$, we can use Equation \ref{eq:v1_to_v2}
to convert it to $\hat{v}_2$. For example, let $\hat{v}_1 = \si{\centi\meter}$ and $\hat{v}_2 = \si{\inch}$.
Then:
\begin{align}
  \si{\centi\meter} &= \SI{0.01}{\meter} \\
  \si{\inch} &= \SI{0.0254}{\meter} 
\end{align}
and
\begin{align}
  \si{\inch} &= 0.0254\times \SI{100}{\centi\meter}  = \SI{2.54}{\centi\meter}
\end{align}
Whic is a conversion \emph{from} \si{\centi\meter} \emph{to} \si{\inch}. But if we want to convert a quantity expressed in \si{\centi\meter}
to \si{\inch}, say \SI{10}{\centi\meter} for example, we would do:
\begin{align}
  \SI{10}{\centi\meter} = \SI{10}{\centi\meter} \frac{\si{\inch}}{\SI{2.54}{\centi\meter}} = \frac{10}{2.54} \si{\inch}
\end{align}
So the numerical value of a quantity expressed in a given unit transforms in the opposite way as the unit. This is similar to the difference
between covariant and contravariant vectors, where the components of contravariant vector transform in the opposte way of the basis vectors.

To convert $x = v_1 \hat{v}_1$ to $\hat{v}_2$, we need to replace $\hat{v}_1$.
\begin{align}
  \hat{v}_1 &= \frac{\alpha_1}{\alpha_2}\hat{v}_2  + \left( \beta_1- \frac{\alpha_1}{\alpha_2}\beta_2 \right) \\
  v_1 \hat{v}_1 &= v_1 \left(\frac{\alpha_1}{\alpha_2}\hat{v}_2  + \left( \beta_1- \frac{\alpha_1}{\alpha_2}\beta_2 \right)\right) \\
  v_1 \hat{v}_1 &= v_1 \frac{\alpha_1}{\alpha_2}\hat{v}_2  + v_1 \left( \beta_1- \frac{\alpha_1}{\alpha_2}\beta_2 \right)
\end{align}
So what is $v_2$? For absolute units, we would just have $v_2 = v_1\frac{\alpha_1}{\alpha_2}$. But here we have the second term
with no unit to deal with. First, consider the cases when of the units is an absolute unit. Let $\hat{v}_1$ be an absolute unit, so
that $\beta_1 = 0$. Then we have
\begin{align}
  v_1 \hat{v}_1 &= v_1 \frac{\alpha_1}{\alpha_2}\hat{v}_2  - v_1 \frac{\alpha_1}{\alpha_2}\beta_2
\end{align}
The $v_1$ in the second term does not make any sense here. If we consider the case when $\hat{v}_1 = \si{\kelvin}$ and $\hat{v}_2 = \si{\celsius}$,
then this says that
\begin{align}
  \SI{100}{\kelvin} &= 100 \si{\celsius}  - 100 \times 273
\end{align}
which is not correct. The correct equation is of course
\begin{align}
  \SI{100}{\kelvin} &= 100 \si{\celsius}  - 273,
\end{align}
but this doesn't even appear to be dimensionally correct. The issue is that units don't obey the normal algebra. It does not make sense to multiply
an offset unit by a scaling factor (what is \SI{100}{\celsius} times 2?), so we cannot do it.

\section{Correct Way}

The correct way to convert offset units is to first subract off the bias,
convert the scale, and add the new bias on.  Note that the $\beta$ term in
Equation \ref{eq:offset_unit} is a numerical value expressed in the \emph{scale
of the unit being defined}. When we convert from the absolute unit, to the offset unit,
the $\beta$ term is \emph{not} scaled. i.e.
\begin{align}
  100\hat{u} \rightarrow \alpha (100\hat{u}) + \beta \ne \alpha (100\hat{u}) + 100 \beta
\end{align}

Note that, while $\hat{v}$ is an offset unit, $\hat{v}-\beta$ is \emph{not}. This
illustrates the breakdown in our notation, it looks like we are adding two different things.

Reformulating our conversion then
\begin{align}
  \hat{v}_1  - \beta_1 &= \alpha_1 \hat{u} \\
  \hat{v}_2  - \beta_2 &= \alpha_2 \hat{u}.
\end{align}
Since $\hat{v}_i  - \beta_i$ is an absolute unit, it will obey our usual algebra, as long as we keep it together. Let $\hat{v}^\prime_i = \hat{v}_i - \beta_i$. Then
\begin{align}
  \hat{v}_1  - \beta_1 &= \alpha_1 \hat{u} = \hat{v}^\prime_1 \\
  \hat{v}_2  - \beta_2 &= \alpha_2 \hat{u} = \hat{v}^\prime_2 ,
\end{align}
and
\begin{align}
  \hat{v}^\prime_1 &= \alpha_1 \hat{u} \\
  \hat{v}^\prime_2 &= \alpha_2 \hat{u} = \frac{\alpha_2}{\alpha_1} \hat{v}^\prime_1 \\
  \hat{v}^\prime_1 &= \frac{\alpha_1}{\alpha_2} \hat{v}^\prime_2 \\
\end{align}
Note that $\frac{\alpha_1}{\alpha_2}$ (expressed in terms of each units scaling factor for the base unit of the dimension)
is the conversion factor from $\hat{v}_2$ to $\hat{v}_1$ but will convert quantities expressed
in $\hat{v}_1$ to $\hat{v}_2$.
$$
v^\prime_1 \hat{v}^\prime_1 = v^\prime_1 \frac{\alpha_1}{\alpha_2} \hat{v}^\prime_2 = v^\prime_2 \hat{v}^\prime_2
$$
Now we just need a rule to convert between quantities expressed in terms of the primed and unprimed units:
\begin{align}
  v \hat{v} &= (v - \beta)\hat{v}^\prime = v^\prime \hat{v}^\prime \\
  v^\prime \hat{v}^\prime &= (v^\prime + \beta)\hat{v} = v\hat{v}
\end{align}
Now given $x = v_1\hat{v}_1$
\begin{align}
  v_1 \hat{v}_1 &= (v_1 - \beta_1) \hat{v}^\prime_1 = (v_1 - \beta_1) \frac{\alpha_1}{\alpha_2} \hat{v}^\prime_2 = \left((v_1 - \beta_1) \frac{\alpha_1}{\alpha_2} + \beta_2\right) \hat{v}_2 \\
  v_2 &= \left((v_1 - \beta_1) \frac{\alpha_1}{\alpha_2} + \beta_2\right)
\end{align}

\subsection{Final Note}

The discussion about covariant and contravariant vectors above is fun, but it does have some practical consequenses. In fact, it is the reason
I started writing this document.

Take the well known conversion forula for Celsius to Fahrenheit:
$$
\si{\fahrenheit} = \frac{9}{5} \si{\celsius} + 32
$$

The Rankine scale is an aboslute temperature scale that uses the Fahrenheit degree size. That is, Rankine is zero at absolute zero, but
the size of one degree is the same as Fahrenheit. So, to convert from Kelvin to Rankine, we have
\begin{equation}
  \label{eq:rankine_quant}
\si{\rankine} = \frac{9}{5} \si{\kelvin}
\end{equation}
Now, since a Kelvin degree is larger than a Rankine degree, we also have
$$
\SI{900}{\rankine} = \SI{500}{\kelvin}
$$
but this means that 
\begin{equation}
  \label{eq:rankine_unit}
\si{\rankine} = \frac{5}{9}\si{\kelvin}.
\end{equation}
So which is correct?

The answer is that they are both correct, but Equation \ref{eq:rankine_quant} is a conversion for the numerical values of the quantity, and
Equation \ref{eq:rankine_unit} is a conversion for the units. Perhaps Equation \ref{eq:rankine_quant} should be written as
\begin{equation}
  T_R = \frac{9}{5} T_K,
\end{equation}
i.e. a relationships between the \emph{components} of a quantity. The point here is that we must be careful when defining unit conversions
in the library. The library expects conversions for the units, \emph{not} the components, so Equation \ref{eq:rankine_unit} is the correct way
to specify the unit.

\section{General Case}

In general, a unit may be defined as an arbitrary transformation. For example, the American Wire Gauge unit is defined as
$$
d_n = \SI{0.127}{\milli\meter} \times 92^{\frac{36 - n}{39}}
$$
where $n$ is the wire gauge. To perform arbitrary unit conversions, including conversions between non-linear units, we just need the
function that relates unit to a linear, absolute base unit. Given two units:
\begin{align}
\hat{v}_1 &= f(\hat{u}) \\
\hat{v}_2 &= g(\hat{u})
\end{align}
%then
%\begin{align}
  %\hat{u} &= f^{-1}(\hat{v}_1) \\
%\hat{v}_2 &= g(f^{-1}(\hat{v}_1)).
%\end{align}
To convert quantities, be just convert to the base unit first, then to the second unit.
\begin{align}
v_1 \hat{v}_1 &= g(f^{-1}(v_1)) \hat{v}_2 = v_2 \hat{v}_2 \\
v_2 &= g(f^{-1}(v_1))
\end{align}



\end{document}

